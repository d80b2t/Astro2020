%%%%%%%%%%%%%%%%%%%%%%%%%%%%%%%%%%%%%%%%%%%%%%%%%%%%%%%%%%%%%%%%%%%%%%%%%%%%%%%%
%%
%%  https://tex.stackexchange.com/questions/337820/mcq-long-table-using-tikz-tcolorbox-or-tabular
%%  https://tex.stackexchange.com/questions/283419/color-in-a-multirow-cell-with-extra-vertical-space/283454
%%  https://tex.stackexchange.com/questions/406033/how-to-fit-a-cell-of-a-table-to-a-figure-and-arrange-multiple-tables/406042
%% 
%% THIS (??)::
%%     https://texblog.org/2014/05/19/coloring-multi-row-tables-in-latex/
%%
%%
%%   https://www.inf.ethz.ch/personal/markusp/teaching/guides/guide-tables.pdf
%%
%%%%%%%%%%%%%%%%%%%%%%%%%%%%%%%%%%%%%%%%%%%%%%%%%%%%%%%%%%%%%%%%%%%%%%%%%%%%%%%%


\begin{tcolorbox}[tab1, tabularx={X  X }, title=Outstanding Issues in Variable Extragalactic Astrophysics, boxrule=1.25pt] 
Scientific Motivation                 &  NEOCam Requirements      \\ 
\hline \hline
\multicolumn{2}{c}{{\sc The physics of accretion}} \\ 
Investigate ``hot'' and ``cold'' mode accretion in the quasar
population; determine the rates and timescales characterising the
Changing Look Quasar (CLQ) population.  &
Identify and characterize all the CLQs in DESI, LSST, WISE and NEOCam footprint.\\
\hline
Probe and determine the physical state of the inner parsec of the
quasar central engine.  & 
Rapid analysis and response for NEOCam quasar light curves. \\
\hline
%%
\multicolumn{2}{c}{{\sc Obscured accretion and galaxy formation}} \\
\hline
%%
Establish the bolometric output and origin of IR emission, and
determine presence of extreme outflows in the quasar population (??!!) & 
Deep optical imaging data from LSST combined with searching for post-starburst
signatures in NEOCam light-curves. \\
\hline
%%
Establishing the range of SED parameter space the quasars occupy by a
multi-wavelength multi-epoch ``truth table dataset''. & 
Build ``The Quasar SED Rosetta Stone'' using X-ray, UV/optical, IR
data as well as repeat optical observations from LSST, NEOCam, spectroscopy. \\ 
\hline
%%
\multicolumn{2}{c}{{\sc Multi-messanger Astrophysics}}\\
Identify the EM counterparts for $\sim10^{6}$ M$_{\odot}$ supermassive binary black holes; 
Identify the EM counterparts for neutrino events. &  
\\ 
    %\end{tcbitemize}
\end{tcolorbox}
