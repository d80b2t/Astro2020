\documentclass[12pt]{article}
\usepackage{times}
\usepackage{geometry}
\geometry{letterpaper, portrait, margin=1in}
\usepackage[utf8]{inputenc}
\usepackage{enumitem,amssymb}
\usepackage{ragged2e}
\usepackage[usenames,dvipsnames]{xcolor}

\newlist{thematic}{itemize}{8}
\setlist[thematic]{label=$\square$}
\usepackage{pifont}
\newcommand{\cmark}{\ding{51}}%
\newcommand{\xmark}{\ding{55}}%
\newcommand{\done}{\rlap{$\square$}{\raisebox{2pt}{\large\hspace{1pt}\cmark}}%
\hspace{-2.5pt}}
\newcommand{\wontfix}{\rlap{$\square$}{\large\hspace{1pt}\xmark}}

\begin{document}
\raggedright
\huge
Astro2020 Science White Paper \linebreak

Replace with Your Title \linebreak
\normalsize

\noindent \textbf{Thematic Areas:} \hspace*{60pt} $\square$ Planetary Systems \hspace*{10pt} $\square$ Star and Planet Formation \hspace*{20pt}\linebreak
$\square$ Formation and Evolution of Compact Objects \hspace*{31pt} $\square$ Cosmology and Fundamental Physics \linebreak
  $\square$  Stars and Stellar Evolution \hspace*{1pt} $\square$ Resolved Stellar Populations and their Environments \hspace*{40pt} \linebreak
  $\square$    Galaxy Evolution   \hspace*{45pt} $\square$             Multi-Messenger Astronomy and Astrophysics \hspace*{65pt} \linebreak
  
\textbf{Principal Author:}

Name:	
 \linebreak						
Institution:  
 \linebreak
Email: 
 \linebreak
Phone:  
 \linebreak
 
\textbf{Co-authors:} (names and institutions)
  \linebreak

\textbf{Abstract  (optional):}


\pagebreak
Insert your white paper text here (max of five pages including figures).

{\bfseries \large \textsc{\textcolor{Cerulean}{Background}}}

This White Paper deeply motivates the {\it Time Domain Extragalactic Science case} for 
the NASA Near-Earth Object Camera (NEOCam). 
%%
NEOCam is a NASA Planetary mission, currently in Phase A, whose goal is to discover and characterize asteroids and comets, to assess the hazard to Earth from near-Earth objects, and to study the origin, evolution, and fate of asteroids and comets.

\smallskip
\smallskip
\noindent
However, as the Infrared Astronomical Satellite (IRAS; in the 1980s, citet), 
the Infrared Space Observatory (ISO, cite) in the mid-1990s, the 
Spitzer Space Telescope (SSC; 2003-ongoing) and 
the Wide-field Infrared Survey Explorer (WISE, 2009-2011; citet) and 
the NEOWiSE-R (2013-ongoing) 
shows how powerful infrared space telescopes, with moderate, $\sim$tens of cm, sized 
primary mirrors are. 

\smallskip
\smallskip
\noindent
In particular, extragalactic science has always benefitted from space-based IR operations, and this will continue in the 2020s with JWST, as well as new very wide-field observatories such as NEOCam. 

Here we highlight 3 particular extragalactic science cases all of which have a time-domain aspect. These are Type Ia SNe; ``Changing Look Quasars'' and IR variable signatures of Gravitational wave events. As such, these range in established techniques via very recent progress to brand new parameter space. 

\smallskip
\smallskip
\noindent
\textbf{\textsc{The world's largest digital camera, billions of objects and  the first motion picture of our universe: }} 

- IR SNe, Type Ias?

\smallskip
\smallskip
\noindent
-- Ross et al., Stern et al. 

\smallskip
\smallskip
\noindent
-- but redder...

\smallskip
\smallskip
\noindent
-- links to `ERQs'

\smallskip
\smallskip
\noindent
-- UFOs in the Seyferts (Hamman et al.?? Dom's object??)

\smallskip
\smallskip
\noindent
-- Full LIGO mode
-- LISA


\smallskip
\smallskip
\noindent
-- ``ROI'' in e.g. Exgal science for just the further data analysis. 

\smallskip
\smallskip
\noindent
NEOCam:: 65 -  120 $\mu$Jy for NC1 and 110-280 $\mu$Jy for NC2, 
WISE:: 0.08, 0.11 for W1/2 mJy, i.e. 80 and 110 $\mu$Jy.

\smallskip
\smallskip
\noindent
Overlapp with JWST??!!

\pagebreak
\textbf{References}



\end{document}

