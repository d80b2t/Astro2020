\documentclass[12pt]{article}


\input{format}


\begin{document}
\raggedright
\huge
Astro2020 Science White Paper \linebreak

Opportunities in Time-Domain Extragalactic Astrophysics with the NASA
Near-Earth Object Camera (NEOCam)
\linebreak
\normalsize

\noindent \textbf{Thematic Areas:} 
\hspace*{60pt} $\square$ Planetary Systems 
\hspace*{10pt} $\square$ Star and Planet Formation 
\hspace*{20pt} 
\linebreak
$\square$ Formation and Evolution of Compact Objects 
\hspace*{31pt} $\square$ Cosmology and Fundamental Physics 
\linebreak
  $\square$  Stars and Stellar Evolution \hspace*{1pt} 
 $\square$ Resolved Stellar Populations and their Environments \hspace*{40pt} 
\linebreak
  $\square$    Galaxy Evolution   \hspace*{45pt} $\square$             
Multi-Messenger Astronomy and Astrophysics \hspace*{65pt} \linebreak
  
\textbf{Principal Author:}

Name: Nicholas P. Ross	
 \linebreak						
Institution:  University of Edinburgh
 \linebreak
Email: npross@roe.ac.uk
 \linebreak
Phone:  +44 (0)131-668 8351
 \linebreak
 
\textbf{Co-authors:} (names and institutions)
  \linebreak
Roberto J. Assef (Universidad Diego Portales)   \linebreak


\textbf{Abstract  (optional):}
This White Paper motivates the {\it time domain extragalactic science
case} for the NASA Near-Earth Object Camera (NEOCam).
NEOCam is a NASA Planetary mission, currently in Phase A, whose goal is to
discover and characterize asteroids and comets, to assess the hazard
to Earth from near-Earth objects, and to study the origin, evolution,
and fate of asteroids and comets.
%% 
NEOCam will, however, cover 68\% of the `extragalactic sky'
%($ |b| > 30$ deg), 
and as the NEOWISE-R mission has recently proved,
infrared information is now vital for identfying and characterizing
the $\gtrsim$10 million IR bright AGN, as well as using the IR light
curve to provide deep insights into accretion disk astrophysics.
%%
NEOWISE-R data has also been used to discover 

As such, for relatively very little additional cost, adding the
capacity for additional NEOCam data processing (and/or alerting) would
have a massive scientific and legacy impact on extragalactic time
domain science.


\pagebreak
%%Insert your white paper text here (max of five pages including figures). 
%\begin{quotation} \noindent {\it   } \noindent \end{quotation} 

\smallskip
\smallskip
\noindent
{\bfseries \textsc{\textcolor{Cerulean}{Time Domain Extragalactic Science Now and into the Net Decade}}}

\smallskip
\noindent
Having started in the 2010s with Pan-STARRS, PTF and ZTF, and with the
arrival of LSST, time domain extragalactic science will be a fully
mature field in the 2020s. However, the mid-infrared extragalactic time domain 
field was completely unexplored until first the {\it Spitzer}
Deep Wide-Field Survey over the NOAO Deep Wide Field Survey Bo\"{o}tes
field, (Koz{\l}owski et al. 2010, 2016) which has now beomce the 
Decadal IRAC Bootes Survey (DIBS; Ashby et al. 2013). 
and then the advent of the Wide-field Infrared
Survey Explorer (WISE) and Near-Earth Object Wide-field Infrared
Survey Explorer Reactivation missions (NEOWISE-R). 

\smallskip
\noindent
The Wide-field Infrared Survey Explorer (WISE) is a NASA
infrared-wavelength astronomical space telescope launched in December
2009, which performed a ``Cold'' mission imaging out to 28$\mu$m 
in four bands centered at wavelengths of 3.4, 4.6, 12, and 23$\mu$m.  
The WISE satellise then performed an extended mission called Near-Earth Object WISE
(NEOWISE) though only in the two shortest bands. The WISE satellite was then placed in hibernation in 2011
Febraury, but was reactivated in December 2013.  Since then, the
NEOWISE-R(eactivation) mission has been mapping the full sky.
NEOWISE-R survey observations are continuing in 2019, and as of
mid-February 2019, NEOWISE is 40\% of the way through its 11th
coverage of the sky since the start of the Reactivation mission.

\smallskip
\noindent
Critically, these 11 epochs of 3-5$\mu$m infrared data have now been
used in conjunction with active galactic nuclei (AGN) and quasar datasets in order to find
large numbers of dramatically varying quasars for the very first time
(e.g. Assef et al. 2018a, Assef et al. 2018b, Stern et al., 2018; Ross et al. 2018).
NEOWISE data has also been critical in detecting dust echos associated 
with Tidal Disruption Events 
(e.g. Dou et al. 2017 etc. etc. etc. here)
%However, due to WP space constraints we wont discuss this much further.


\smallskip
\noindent
Because NEOWISE-R is an asteroid-characterization and discovery project, the mission itself does not publish any coadded data products of the sort that maximize the raw NEOWISE data’s value for extragalactic (time-domain) astrophysics. However, after the initial work of Lang, Schlegel and Hogg (e.g. Lang, D. 2014, AJ, 147, 108; Lang, D., Hogg, D. W., \& Schlegel, D. J. 2014, arXiv:1410.7397), Schlafly, Meisner and Green have lead a wide-ranging effort to repurpose NEOWISE observations for astrophysics, starting by building deep full-sky coadds from tens of millions of 3.4$\mu$m (W1) and 4.6$\mu$m (W2) exposures (e.g. 2019, ApJS, 240, 30 and references therein). 

\smallskip
\noindent
The resulting ``unWISE'' line of data products, we have already created the deepest ever full-sky maps at 3–5$\mu$m 
(Meisner et al. 2017a, AJ, 153, 38;  
Meisner et al. 2017b, AJ, 154, 161; 
Meisner et al., 2018a, RNAAS, 2, 1), 
generated a new class of time-domain WISE coadds 
(Meisner et al.  2018b, AJ, 156, 69; 
Meisner et al.  2018c, RNAAS, 2, 202), and performed forced photometry on these custom WISE coadds at the locations of more than a billion optically detected sources
 (Lang et al. 2014; Schlegel et al. 2015; Dey et al. 2018).

\smallskip
\noindent
The NASA Near-Earth Object Camera (NEOCam\footnote{\href{https://neocam.ipac.caltech.edu/}{https://neocam.ipac.caltech.edu/}}) is a NASA Planetary mission, currently in Phase A, whose goal is to discover and characterize asteroids and comets carrying on the legacy of NEOWISE.
%
{\it The main purpose of this White Paper is to encourage, in the strongest terms, NASA Astrophysics to 
budget and plan for a line of e.g. coadded data products from the NEOCam mission that will result in a substantial extragalactic resource. As NEOWISE-R and unWISE have proved, the ``Return On Investment'' here is considerable.}


\smallskip
\smallskip
\noindent
%{\bfseries \large \textsc{\textcolor{Cerulean}{Continuing the NASA IR Space Telescope Legacy}}}
{\bfseries \textsc{\textcolor{Cerulean}{Continuing the NASA IR Space Telescope Legacy}}} 

\smallskip
\noindent
As the Infrared Astronomical Satellite (IRAS; 1983 January 25th launch, Wheelock et al. 1994), 
%the Infrared Space Observatory (ISO, cite) in the mid-1990s, 
the Spitzer Space Telescope (SSC; 2003 August 25th launch) and the Wide-field
Infrared Survey Explorer (WISE; 2009 December 14th launch) satellites all show 
how powerful infrared space telescopes, with moderate-sized primary mirrors are.

\smallskip
\smallskip
\noindent
NEOCam would survey the Solar System at two simultaneous thermal IR bandpasses. 
%NEOCam would detect and track the majority of potentially hazardous asteroids during its planned 5-year lifetime, and constrain the population of smaller objects that could pose a threat to Earth. NEOCam 
but would also provide a synoptic survey of two-thirds of the thermal infrared sky at 4 - 5.2$\mu$m and  6 - 10$\mu$m using a large field of view of 11.56 square degrees. 
%We will describe the mission as proposed, and the expected data products.



\smallskip
\smallskip
\noindent
{\bfseries \textsc{\textcolor{Cerulean}{The NASA Near-Earth Object Camera (NEOCam)}}}

\textsl{\textsc{Overview:}}
NEOCam is a NASA Planetary mission, currently in Phase A, whose goal
is to discover and characterize asteroids and comets, to assess the
hazard to Earth from near-Earth objects, and to study the origin,
evolution, and fate of asteroids and comets. NEOCam is a
single-instrument, 50cm diameter telescope that will observe in two
simultaneous channels, called NC1 and NC2, which cover wavelengths of
4.0-5.2 $\mu$m and 6.0-10.0$\mu$m, respectively. There are four Teledyne
2k$\times$2K HgCdTe arrays per channel with 3.00'' pixels in each. The entire
celestial sphere between ecliptic latitudes of +40$^{\circ}$ to -40$^{\circ}$ will
continually be mapped over the course of the 5-yr mission, and over 12
yr for the design lifetime. 
%%
{\it NEOCam will provide a synoptic survey of two-thirds of the thermal infrared sky at 4-5.2 microns and 6-10 microns.}

\smallskip
\smallskip
\noindent
\textsl{\textsc{Cadence:}} Estimates of on-board overheads and resulting orbit quality are still being refined, but the current data cadence is as follows. The observing pattern consists of a four-peat of a six-position dither (a quick sequence of six images with 28s integration time each) with 2h gaps between each repeat. This four-peat will recur $\sim$13.2d later as long as the 75d visibility window is still open. Afterward, there will be a gap of 215d until the next visibility window opens and the pattern begins again. On average in a typical visibility window, there are $\sim$23 of these six-position dither sequences (a little less than 6 four-peats), $\sim$234 over 5yr, and $\sim$562 over 12yr.

\smallskip
\smallskip
\noindent
\textsl{\textsc{Depth:}} Each of the six-position dither sequences is expected to have an S/N=5 sensitivity of 65-120 $\mu$Jy for NC1 and 110-280 $\mu$Jy for NC2, for low to high zodiacal backgrounds. 

\smallskip
\smallskip
\noindent
\textsl{\textsc{Data Products:}} NEOCam processing will create images and lists of characterized sources from each individual exposure and each stacked six-dither position sequence. It will also create differenced images by subtracting a static reference image, and a list of the characterized transient detections also produced. Because the goal of NEOCam is to provide and characterize moving objects within the solar system, coadd and source extractions over longer timescales are not provided, the one exception being yearly builds to create new static images (without any source detection or characterization) of the sky to use in image differencing. No alerting mechanism is provided for astrophysical transient events.  

\smallskip
\smallskip
\noindent
{\bf To realize the full potential of the NEOCam data for astrophysical research, additional data products and alerting infrastructure are needed, as discussed below. For a relatively small investment, NASA can leverage the existing NEOCam data to cover a wide range of extragalactic, time-domain, research.}


\smallskip
\smallskip
\noindent
In particular, extragalactic science has always benefitted from space-based IR operations, and this will continue in the 2020s with JWST, as well as new very wide-field observatories such as NEOCam. 
%%
Here we highlight 3 particular extragalactic science cases all of which have a time-domain aspect. These are Type Ia SNe; ``Changing Look Quasars'' and IR variable signatures of Gravitational wave events. As such, these range in established techniques via very recent progress to brand new parameter space. 

\smallskip
\smallskip
\noindent
{\bfseries \textsc{\textcolor{Cerulean}{
Near-infrared Variability: A cornerstone of AGN and Quasar investigations in the next Decade and beyond
}}}


\smallskip
\smallskip
\noindent
%% From:: 
%%   The Central Engine of Active Galactic Nuclei
%%   ASP Conference Series, Vol. 373, Xi’an, China, 16–21 October, 2006 eds. L. C. Ho and J.-M. Wang
%%   Dust in Active Galactic Nuclei
%%   Aigen Li

The circumnuclear dust absorbs the AGN illumination and reradiates the absorbed energy in the IR. The IR emission at wavelengths longward of $\lambda > 1 \mu$m accounts for at least 50\% of the bolometric luminosity of type 2 AGNs. For type 1 AGNs, $\sim$10\% of the bolometric luminosity is emitted in the IR (e.g. see Fig. 13.7 of Osterbrock \& Ferland 2006). A near-IR ``bump'' (excess emission above the $\sim$2–10$\mu$m continuum), generally attributed to hot dust with temperatures around $\sim$1200– 1500K (near the sublimation temperatures of silicate and graphite grains), is seen in a few type 1 AGNs (Barvainis 1987; Rodriguez-Ardila \& Mazzalay 2006).

\smallskip
\smallskip
\noindent
As discussed in detail in Stern et al. (2005, ApJ, 631, 163), the lack of strong polycyclic aromatic hydrocarbon emission in powerful AGNs, along with the IR flux $\lambda_{\rm rest} < 5 \mu$m flux of AGNs being dominated by power-law emission rather than a composite stellar spectrum, leads to the  AGNs being significantly redder than t
hat of lower-redshift galaxies. 

\smallskip
\smallskip
\noindent
\textbf{\textsc{Near-infrared Variability Case Study: Super-Luminous Supernova in AGN torus}} 

\smallskip
\smallskip
\noindent
Assef et al. (2013, ApJ, 772, 26; 2018, ApJS, 234, 23) have led the field in identifying large AGN candidate samples from selected from the Wide-field Infrared Survey Explorer (WISE) observations. 
%%
From their most reliable study, Assef et al. (2018, ApJ, 866, 26) present spectroscopic observations of some of the most infrared variable extragalactic candidates to constrain their nature. {\it They find that from a sample of 45 objects with strong IR variability, only seven show significant optical variability.}  

\smallskip
\smallskip
\noindent
Further investigations reveals that one of these objects, WISEA J094806.56+031801.7  is most likely a super-luminous supernova (SLSN) with total radiated energy to be $E=1.6\pm0.3 \times 10^{52}$ erg, {\it making it one of the most energetic SLSNe observed.} Based on the lack of change in mid-IR color throughout and after the transient event, the speculation is that the location of the SLSN is within the torus of the AGN. 


\smallskip
\smallskip
\noindent
\textbf{\textsc{Investigation AGN and quasar central Engines via the ``Changing Look'' phenomenon:}}

\smallskip
\smallskip
\noindent
``Changing-Look quasars'' 
 in which the strong UV continuum and broad hydrogen emission lines associated with unobscured quasars either appear or disappear on observed-frame timescales of months-years (e.g., LaMassa et al. 2015; Macleod et al. 2016; Ruan et al. 2016a, 2016b; Runnoe et al. 2016; Gezari et al. 2017; Yang et al. 2018). 
%%
CLQs are important since they offer a direct observational probe into the physical processes dictating the structure of the broad-line region (BLR), the accretion disk and likely the innermost circular orbit (ISCO), 
and do so on human (and indeed, $\sim$postdoc career) timescales. 
%%
These timescales can potentially be associated with the viscous timescale (drift time through the accretion disk), the light crossing timescale (critical for reverberation mapping and disk reprocessing) and the dynamical timescale of the BLR.  {\it CLQs are thus an ideal laboratory for studying accretion physics, as the entire system responds to a large change in ionizing flux on a human timescale}. 
%%
The physical processes responsible for these changing-look quasars are still debated, but physical changes in the accretion disk structure appear to be the more likely cause rather than changes in obscuration. However, this leads to a irreconcilable breakdown of the classical, thin accretion disk model, leading to what has recently been called the ``Quasar Viscosity Crisis'' (Lawrence, 2018, Nature Astronomy). 

\smallskip
\smallskip
\noindent
\textbf{\textsc{New IR investigations into the CLQ Population:}}
However, we have begun to make progress with new
observational experiments and tests.  Taking advantage of new IR
light-curves from NEOWISE-R ([26, 27]), we have begun to make in-roads
into understanding the CLQ population.  This includes [28, 29]
identifying objects with rapidly changing IR light-curves and also
accretion disk changes, e.g. the $z=0.378$ quasar SDSS
J110057-005304.4, 
%see Figure~\ref{fig:J110057}. 
From J1100-0053, a new model ([28]) suggests a dramatic new picture of the physics of the
CLQs governed by processes at the ISCO and the structure of the
innermost disk.  

%{\it We have embarked on a new observation campaign gaining optical light-curves (from the Liverpool Telescope and the Zwicky Transient Facility) and spectra (from WHT and Palomar) with data already in hand from 2018A and 2018B, to test this startling new hypothesis.}

%-- Ross et al., Stern et al. \\
%-- but redder...\\

\smallskip
\smallskip
\noindent
Without any changes to te scienfitic requirements, or mission profile, NEOCam offers 
a dramatic new dataset for extragalactic time domain astrophysics investigtaions, including
the long term monitoring of the CLQ population that we are identifying today.
%%
{\it With infrared emission from AGN associated with structures $\sim$a few light months to $\sim$several light years from the central engine accretion disk and photon source, variable AGN observed in the UV/optical during the early/mid 2020s from LSST will be NEOCam extragalactic time domain sources.}


\smallskip
\smallskip
\noindent
\textbf{\textsc{Near-infrared Variability Case Study: Super-Luminous Supernova in AGN torus}} 
From Dou et al. 2016, ApJ, 832, 188:: (!!!) 
The sporadic accretion following the tidal disruption of a star by a
super-massive black hole (TDE) leads to a bright UV and soft X-ray
flare in the galactic nucleus. The gas and dust surrounding the black
hole responses to such a flare with an echo in emission lines and
infrared emission.

\smallskip
\smallskip
\noindent
\textbf{\textsc{Very High redshift Quasars:}} 
Very high redshift quasars (VH$z$Q; defined here to have redshifts $z\geq5.00$) are excellent probes of the early Universe. This includes studies of the Epoch of Reionization for hydrogen (see e.g., Fan2006, Mortlock2016 for reviews), the formation and build-up of supermassive black holes (e.g., Rees1984, WyitheLoeb2003, Volonteri2010, Agarwal2016, Valiante2018, Latif2018, Wise2019) and early metal enrichment (see e.g., Simcoe2012, Chen2017, Bosman2017).

\smallskip
\smallskip
\noindent
Wang et al (2018) 
report the discovery of 16 quasars at $6.4\lesssim z \lesssim 6.9$ 
with Chapman \& Shanks (2019) reporting two further new $z\gtrsim6$ quasars. 
Using DES, VHS and unWISE, Yang et al. (2018) report the discovery of six new luminous quasars at $z > 6.4$, 
including an object at $z = 7.02$, only the fourth quasar yet known at $z > 7$. 

\smallskip
\smallskip
\noindent
Recently, Ross \& Cross (2019) have compiled a database of all 
spectroscoipcially confirmed $z\geq5.00$ quasars, totalling 463 objects (as of 
2018-Dec-31). Of these 463 quasars, 283 are detected in the WISE ``ALLWISE'' W1/2 
data release. {\it However, this detection rate increases to 362 in W1 and 308 in W2 
in the unWISE catalogs, i.e. an increase in detection rate by 28\% and 9\%, in W1 and W2, 
respectively (the equivalent to a 5.00 year mission being run for 6.4 and 5.45 years).}

\smallskip
\smallskip
\noindent
{\it Thus the impact of NEOWISE-R and NEOCam on the Physics of the Cosmos and 
Cosmic Origins programs is direct and considerable.}

\smallskip
\smallskip
\noindent
\textbf{\textsc{Connecting AGN activity in the IR with multi-messenger blazars: }}
With the electromagnetic association with GW1708017 (e.g., Abbott et
al. 2017a,b,c, Cowperthwaite et al. 2017, Soares-Santos, et
al. 2017)
%%
and the neutrino emission from the direction of the blazar TXS 0506+056 
(IceCube Collaboration; 2018a,b).
we have  fully entred the era of ``multi-messenger astronomy'' 

The 3rd Generation Ground-based Gravitational-wave Observatory Network
(``3G'') and the ESA-NASA Laser Interferometer Space Antenna (LISA). 

\smallskip
\smallskip
\noindent
%%%%%%%%%%%%%%%%%%%%%%%%%%%%%%%%%%%%%%%%%%%%%%%%%%%%%%%%%%%%%%%%%%%%%%%%%%%%%%%%
%%
%%  https://tex.stackexchange.com/questions/337820/mcq-long-table-using-tikz-tcolorbox-or-tabular
%%  https://tex.stackexchange.com/questions/283419/color-in-a-multirow-cell-with-extra-vertical-space/283454
%%  https://tex.stackexchange.com/questions/406033/how-to-fit-a-cell-of-a-table-to-a-figure-and-arrange-multiple-tables/406042
%% 
%% THIS (??)::
%%     https://texblog.org/2014/05/19/coloring-multi-row-tables-in-latex/
%%
%%
%%   https://www.inf.ethz.ch/personal/markusp/teaching/guides/guide-tables.pdf
%%
%%%%%%%%%%%%%%%%%%%%%%%%%%%%%%%%%%%%%%%%%%%%%%%%%%%%%%%%%%%%%%%%%%%%%%%%%%%%%%%%


\begin{tcolorbox}[tab1, tabularx={X  X }, title=Outstanding Issues in Variable Extragalactic Astrophysics, boxrule=1.25pt] 
Scientific Motivation                 &  NEOCam Requirements      \\ 
\hline \hline
\multicolumn{2}{c}{{\sc The physics of accretion}} \\ 
Investigate ``hot'' and ``cold'' mode accretion in the quasar
population; determine the rates and timescales characterising the
Changing Look Quasar (CLQ) population.  &
Identify and characterize all the CLQs in DESI, LSST, WISE and NEOCam footprint.\\
\hline
Probe and determine the physical state of the inner parsec of the
quasar central engine.  & 
Rapid analysis and response for NEOCam quasar light curves. \\
\hline
%%
\multicolumn{2}{c}{{\sc Obscured accretion and galaxy formation}} \\
\hline
%%
Establish the bolometric output and origin of IR emission, and
determine presence of extreme outflows in the quasar population (??!!) & 
Deep optical imaging data from LSST combined with searching for post-starburst
signatures in NEOCam light-curves. \\
\hline
%%
Establishing the range of SED parameter space the quasars occupy by a
multi-wavelength multi-epoch ``truth table dataset''. & 
Build ``The Quasar SED Rosetta Stone'' using X-ray, UV/optical, IR
data as well as repeat optical observations from LSST, NEOCam, spectroscopy. \\ 
\hline
%%
\multicolumn{2}{c}{{\sc Multi-messanger Astrophysics}}\\
Identify the EM counterparts for $\sim10^{6}$ M$_{\odot}$ supermassive binary black holes; 
Identify the EM counterparts for neutrino events. &  
\\ 
    %\end{tcbitemize}
\end{tcolorbox}





\pagebreak
\textbf{References}
Abbott et al., 2017a,  PhRvL, 119p1101A	\\
Abbott et al., 2017b, ApJ, 848, L12	\\
Abbott et al., 2017c, ApJ, 848, L13	\\
Agarwal et al., 2016 \\
Ashby et al., 2013, sptz.prop10088A \\
Assef et al., 2017, \\
Assef et al., 2018a, \\
Assef et al., 2018b, \\
Fan  et al., 2006, \\
Chapman \& Shanks, 2019, MNRAS, {\it in prep.} \\
Cowperthwaite et al., 2017,  ApJ, 848, L17	\\
IceCube Collaboration, et al., 2018, Science, 361, 146 \\ 
IceCube Collaboration, et al., 2018, Science, 361, 147 \\
Koz{\l}owski et al., 2010, ApJ, 716, 530 \\  
Koz{\l}owski et al., 2016, ApJ, 817, 119 \\
Mortlock  et al., 2016 \\
Soares-Santos, et al. 2017, ApJ, 848, L16	\\
Rees  et al., 1984 \\
Ross et al., 2018, MNRAS, \\
Ross \& Cross, 2019, MNRAS, {\it in prep \href{https://github.com/d80b2t/VHzQ}{draft here}} \\
Volonteri  et al., 2010
Wyithe \& Loeb et al., 2003
Wang et al., 2018,  arXiv:1810.11926v1 \\
Yang et al, 2018, arXiv:1811.11915v1 \\
\end{document}

